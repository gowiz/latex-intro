%\documentclass[13pt,handout]{beamer}
\documentclass[13pt]{beamer}
%\usetheme{Boadilla}

\usepackage[utf8]{inputenc}
\usepackage{tikz}
\usetikzlibrary{arrows,shapes}
\usepackage{graphicx}
\usepackage{gensymb}
\usepackage{verbatim}
\usepackage{multicol}
\usepackage[dutch]{babel}
\usepackage{dot2texi}
\usepackage{hyperref}
\usepackage{xpatch}
\usepackage{xcolor}
\usepackage{realboxes}
\usepackage[T1]{fontenc}
\usepackage{lmodern}

\usepackage{listings}
\lstset{
  language=[LaTeX]TeX,
  frame=single,
  framerule=0pt,
  backgroundcolor=\color{lightgray},
  basicstyle=\small
}
\makeatletter
\xpretocmd\lstinline{\Colorbox{lightgray}\bgroup\appto\lst@DeInit{\egroup}}{}{}
\makeatother

\usepackage{tikz}
\usetikzlibrary{decorations.pathreplacing,calc}
\newcommand{\tikzmark}[1]{\tikz[overlay,remember picture] \node (#1) {};}

\newcommand*{\AddNote}[4]{%
    \begin{tikzpicture}[overlay, remember picture]
        \draw [decoration={brace,amplitude=0.3em},decorate,ultra thick,red]
            ($(#3)!([yshift=1.5ex]#1)!($(#3)-(0,1)$)$) --
            ($(#3)!(#2)!($(#3)-(0,1)$)$)
                node [align=center, text width=2.5cm, pos=0.5, anchor=west] {\tiny #4};
    \end{tikzpicture}
}%

% Display number like Don
\newdimen\dx
\def\dm#1#2{\dx=1em\relax\def\rt{#1}{\Dm#2\dm}}
\def\Dm#1{\ifx#1\dm\else\fontsize\dx\dx\selectfont#1\dx=\rt\dx\expandafter\Dm\fi}

% Set margins first argument left, second argument right
\def\changemargin#1#2{\list{}{\rightmargin#2\leftmargin#1}\item[]}
\let\endchangemargin=\endlist%

\title{\LaTeX introductie}
\subtitle{Typesetting voor leerkrachten secundair onderwijs}
\author{}
\institute{}
\date{}%\today}

\begin{document}

\begin{frame}
  \titlepage%
\end{frame}

\begin{frame}
  \frametitle{Outline}
  \tableofcontents
\end{frame}

\begin{frame}
  \frametitle{Latex}
  \begin{itemize}
  \item Typesetting systeem: genereren van documenten
  \item Nadruk op wetenschappelijke en technische documenten
  \item Schrijver kan zich concentreren op de inhoud
  \item Schrijver kan de lay-out negeren
  \item Verplicht gebruik bij wetenschappelijke publicaties
  \item Beschikbaar op alle systemen: Mac, Linux, Windows, Web, ...
  \item Gratis software en code beschikbaar
  \end{itemize}
\end{frame}

\begin{frame}
  \frametitle{WYSIWYG vs SYSIWYM}
  \begin{columns}[T] % align columns
    \begin{column}{.48\textwidth}
      WYSIWYG
      \begin{itemize}
      \item {\scriptsize What You See Is What You} Get
      \item Tekstverwerkers
      \item Vb: Microsoft Word, Google Docs
      \item Gebruiksvriendelijk
      \item Veel manueel 'prutsen'
      \item Bestandsformaat is gesloten
      \end{itemize}
    \end{column}
    \hfill
    \begin{column}{.48\textwidth}
      WYSIWYM
      \begin{itemize}
      \item {\scriptsize What You See Is What You} Mean
      \item Markup talen
      \item Vb: HTML, Markdown, \TeX, \LaTeX
      \item Steile leercurve
      \item Automatische bladindeling
      \item Brondocument is een tekstbestand
        \begin{itemize}
        \item Altijd leesbaar
        \item Versie controle
        \item Andere tools
        \end{itemize}
      \end{itemize}
    \end{column}%
  \end{columns}
\end{frame}

\begin{frame}
  \frametitle{Geschiedenis van \LaTeX}
  \begin{itemize}
  \item 1436: Gutenberg vindt de boekdrukkunst uit
  \item 1968: Eerste versie van TAOCP, typeset m.b.v. loodzetten
  \item 1976: Tweede versie van TAOCP, typeset m.b.v. fototypesetting
  \item 1978: Knuth neemt één jaar sabbatical om \TeX{} te schrijven
  \item 1980: LaTeX (Leslie Lamport)
  \item 1994: LaTeX 2e (Frank Mittelbach, Chris Rowley, Rainer Schöpf)
  \item 202?: LaTeX 3e
  \end{itemize}

  Versie van \TeX: \dm{0.95}{3.14159265}\\%358979323846264338327950288419716939937510
  En de versie wordt ooit gelijk aan $\pi$
  \begin{itemize}
  \item Niet $t\to+\infty$
  \item Foutloos, stabiel
  \end{itemize}
\end{frame}

\appendix
\section{Editor/Engine/Format/Distribution}

\begin{frame}
  \frametitle{Editor/Engine/Format/Distribution}
  \begin{description}
  \item[Editor] Bewerking van \texttt{.tex} bestanden.
  \item[Engine] Programma die het eigenlijke werk doet, bv. \TeX, pdfTeX, LuaTeX.
  \item[Format] Set macro's die het werken met \TeX vereenvoudigen, bv. \LaTeX, ConTeXt, plainTeX. Dit kunnen ook een set macro's zijn die aangeleverd worden door publisher of onderwijsinstelling.
  \item[Distribution] Reeks extra packages, bijvoorbeeld om met kleur, hyperlinks, chemiesymbolen, enzoverder te werken worden samengepakt, bv. \TeX Live, MiKTeX.
  \end{description}
\end{frame}

\begin{frame}
  \frametitle{Einde}
  \vfill
  \begin{center}
    \huge Vragen?\\
    \Huge Bedankt!
  \end{center}
  \vfill
\end{frame}

\begin{frame}[fragile]
  \frametitle{\LaTeX\ via het web}
  Overleaf: Online \LaTeX\ editor
  \begin{itemize}
  \item Link: \url{https://www.overleaf}.
  \item Inloggen, bijvoorbeeld met gmail account van school.
  \item New project $\to$ Blank project.
  \item Indeling: Bestanden, Input, Output.
  \end{itemize}
  \begin{lstlisting}[mathescape]
    \documentclass{article}         $\tikzmark{lst-begin-preamble}$

    \usepackage[utf8]{inputenc}
    \usepackage[dutch]{babel}
                                    $\tikzmark{lst-end-preamble}$
    \begin{document}                $\tikzmark{lst-begin-document}$

    Mijn eerste stappen in \LaTeX.

    \end{document}                  $\tikzmark{lst-end-document}$
  \end{lstlisting}
  \AddNote{lst-begin-preamble}{lst-end-preamble}{lst-begin-preamble}{Preamble}
  \AddNote{lst-begin-document}{lst-end-document}{lst-begin-document}{Document}
\end{frame}

\begin{frame}[fragile]
  \frametitle{Preamble: Documentclass}
  \begin{itemize}
  \item Soort document, bv. \lstinline{article}, \lstinline{letter}, \lstinline{book}.
  \item Opties:
    \begin{itemize}
    \item Grootte van het lettertype, bv. \lstinline{12pt}.
    \item Soort papier, bv. \lstinline{a4paper}.
    \item Extra opties, bv. \lstinline{twoside}.
    \end{itemize}
  \end{itemize}
  Voorbeelden:
  \begin{lstlisting}
    \documentclass[12pt,a4paper,twoside]{article}
  \end{lstlisting}
  \begin{lstlisting}
    \documentclass[a4paper]{examen}
  \end{lstlisting}
  \begin{lstlisting}
    \documentclass[a3paper]{poster}
  \end{lstlisting}
\end{frame}

\end{document}
